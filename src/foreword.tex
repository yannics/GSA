\frontmatter
\def\thepage{\arabic{page}}
\setcounter{page}{3}

\chapter*{Foreword}
\thispagestyle{empty}
\addcontentsline{toc}{chapter}{Foreword}

\section*{About this edition}
\label{ate}

This edition is a snapshot of an ongoing work. Therefore, it is not an end object. Anyway, it will probably never be a finished product.

\smallskip

Mainly written in English, few sections are in French because there were originally written in French, and for now it is not meaningful to translate it. 

\smallskip

It might remains some correction(s) to be done, some point(s) to be clarify or commented. 

In any case, the reader is invited to formulate any feedback about this book for a future edition.

\smallskip

If so, feel free to contact the author by email: 

\smallskip

\quad \quad \href{mailto:jby.cmsc@gmail.com}{\texttt{by.cmsc@gmail.com}}

\bigskip

The first part of this book regroups articles as chapters fitting my research interest in music analysis and synthesis in term of algorithms. This remains a work in progress, so these articles can evolve in time according to my progress and to the experience of these works \textit{in situ}. 

\smallskip

Most of my works are connected between them, as a development, as a continuation or as an experiment.
%so I do sometimes some reference to some other of my works inside and outside of this book.

\smallskip
For instance, one of the connected work is the \textsl{Neuromuse3} project (N3) which aims to develop algorithms in terms of neuroscience and cognition in a `musical' context. Also a work in progress, the documentation is available on:

\href{https://www.overleaf.com/read/wswcpgqntjrc}{\texttt{\small https://www.overleaf.com/read/wswcpgqntjrc}}

\bigskip
Common Lisp libraries required and designed for this work:

\smallskip

\noindent $\rightarrow$ M2T mentioned in the chapter \textsl{\nameref{m2t}}.

\href{https://github.com/yannics/M2T}{\texttt{\small https://github.com/yannics/M2T}}

\smallskip

\noindent $\rightarrow$ MDS mentioned in the chapter  \textsl{\nameref{mds}}.

\href{https://github.com/yannics/MDS}{\texttt{\small https://github.com/yannics/MDS}}

\smallskip

\noindent $\rightarrow$ \textsl{cl-cycle} mentioned in the chapters \textsl{\nameref{imp1}} and  \textsl{\nameref{imp2}}. 

\href{https://github.com/yannics/cl-cycle}{\texttt{\small https://github.com/yannics/cl-cycle}}

%\bigskip
\newpage
The second and the third parts refer to the \textit{praxis} as concepts and as experiments \textit{in situ}. 

\bigskip

For any reference to this writing, please cite in press as: 

\noindent Yann Ics [2014/\the\year]. \textit{Journal of Generative Sonic Art}. [online] Available at: \href{https://www.overleaf.com/read/sjhfhthgkgdj}{\texttt{\small https://www.overleaf.com/read/sjhfhthgkgdj}} [Accessed \today].

\section*{Third-party softwares}
\label{tps}
%\addcontentsline{toc}{section}{\nameref{tps}}

Some third-party softwares are required for some of these works. 

\subsection*{Praat}

Praat is a free software for the analysis of speech in phonetics. It was designed, and continues to be developed, by Paul Boersma and David Weenink of the University of Amsterdam. 

\href{http://www.fon.hum.uva.nl/praat}{\texttt{\small http://www.fon.hum.uva.nl/praat}}

\bigskip

Praat is used for the analysis of the sound file. Praat can be scripted and it can be called in a Bash script. The Praat manual displays all the needed information concerning the algorithms and the parameters used in Praat.

\subsection*{SBCL}

SBCL means Steel Bank Common Lisp. It is a free software and a mostly-conforming implementation of the ANSI Common Lisp standard.

\href{http://www.sbcl.org}{\texttt{\small http://www.sbcl.org}}

\bigskip

Common Lisp is a very efficient and flexible programming language specifically and historically dedicated to Artificial Intelligence.
SBCL can be called in a Bash script.

\subsection*{Morphologie}

Morphologie is an OpenMusic and PWGL library implemented in Common Lisp. Morphologie is a set of functions of analysis, recognition, classification and reconstitution of symbolic and digital sequences, developed by Jacopo Baboni-Schilingi and Fr\'ed\'eric Voisin at the IRCAM in 1997.

\href{http://www.baboni-schilingi.com/index.php/research}{\texttt{\small http://www.baboni-schilingi.com/index.php/research}}

%\href{http://www.fredvoisin.com/spip.php?article28}{\texttt{\small http://www.fredvoisin.com/spip.php?article28}}

\bigskip

In this work, I had adapted some functions for a 'pure' Common Lisp use.

\subsection*{Midi}

This is a Common Lisp library for parsing MIDI (Musical Instrument Digital Interface) file format files and representing MIDI events.

\href{http://www.doc.gold.ac.uk/isms/lisp/midi}{\texttt{\small http://www.doc.gold.ac.uk/isms/lisp/midi}}

\bigskip

In the chapter \textsl{\nameref{mds}}, section \textsl{\nameref{score}}, the midi package is required as a dependency of M2T.

\subsection*{FluidSynth}

FluidSynth is a free software synthesizer and can be used as a command line shell.

\href{http://www.fluidsynth.org}{\texttt{\small http://www.fluidsynth.org}}

\bigskip

With the package \texttt{fluid-soundfont-gm}, \texttt{FluidSynth} allows converting midi file to a sound file -- see chapter \textsl{\nameref{mds}}, section \textsl{\nameref{cmftmds}}.

\subsection*{SuperCollider}

SuperCollider is a programming language for real time audio synthesis and algorithmic composition.

\href{http://supercollider.github.io}{\texttt{\small http://supercollider.github.io}}

\bigskip

\noindent And also,

\subsection*{SoX}

SoX (Sound eXchange) is a cross-platform command line utility that allows manipulating audio files.

\href{http://sox.sourceforge.net}{\texttt{\small http://sox.sourceforge.net}}

\bigskip

The associated command line SoXI (Sound eXchange Information) allows displaying sound file metadata.

\subsection*{Gnuplot}

Gnuplot is a portable command-line driven graphing utility as an interactive plotting program.

\href{http://www.gnuplot.info}{\texttt{\small http://www.gnuplot.info}}

\subsection*{Graphviz}

Graphviz is open source graph visualization software using graph layout programs, such as Neato -- see note \fullref{neato}.

\href{http://graphviz.org}{\small{\texttt{http://graphviz.org}}}