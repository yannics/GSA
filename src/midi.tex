\providecommand{\tightlist}{%
  \setlength{\itemsep}{0pt}\setlength{\parskip}{0pt}}
\renewcommand{\thesubsection}{}


%\hypertarget{outline-of-the-standard-midi-file-structure}{%
%\section{Outline of the Standard MIDI File
%Structure}\label{outline-of-the-standard-midi-file-structure}}

Go to: {[} \protect\hyperlink{header_chunk}{header chunk} \textbar{}
\protect\hyperlink{track_chunk}{track chunk} \textbar{}
\protect\hyperlink{track_event}{track event} \textbar{}
\protect\hyperlink{meta_event}{meta event} \textbar{}
\protect\hyperlink{sysex_event}{system exclusive event} \textbar{}
\protect\hyperlink{variable_length}{variable length values} {]}

\begin{center}\rule{0.5\linewidth}{0.5pt}\end{center}

A standard MIDI file is composed of "chunks". It starts with a header
chunk and is followed by one or more track chunks. The header chunk
contains data that pertains to the overall file. Each track chunk
defines a logical track.

\begin{verbatim}
 
SMF = <header_chunk> + <track_chunk> [+ <track_chunk> ...]
\end{verbatim}

A chunk always has three components, similar to Microsoft RIFF files
(the only difference is that SMF files are big-endian, while RIFF files
are usually little-endian). The three parts to each chunk are:

\begin{enumerate}
\tightlist
\item
  The track ID string which is four charcters long. For example, header
  chunk IDs are "\texttt{MThd}", and Track chunk IDs are
  "\texttt{MTrk}".
\item
  next is a four-byte unsigned value that specifies the number of bytes
  in the data section of the track (part 3).
\item
  finally comes the data section of the chunk. The size of the data is
  specified in the length field which follows the chunk ID (part 2).
\end{enumerate}

\begin{center}\rule{0.5\linewidth}{0.5pt}\end{center}

\protect\hypertarget{header_chunk}{}{}

\hypertarget{header-chunk}{%
\subsection*{Header Chunk}\label{header-chunk}}

The header chunk consists of a literal string denoting the header, a
length indicator, the format of the MIDI file, the number of tracks in
the file, and a timing value specifying delta time units. Numbers larger
than one byte are placed most significant byte first.

\begin{verbatim}
 
header_chunk = "MThd" + <header_length> + <format> + <n> + <division>
 
\end{verbatim}

\begin{description}
\tightlist
\item[\textbf{\texttt{\ "MThd"}} 4 bytes]
the literal string MThd, or in hexadecimal notation: 0x4d546864. These
four characters at the start of the MIDI file indicate that this
\emph{is} a MIDI file.
\item[\textbf{\texttt{\textless{}header\_length\textgreater{}}} 4 bytes]
length of the header chunk (always 6 bytes long-\/-the size of the next
three fields which are considered the header chunk).
\item[\textbf{\texttt{\textless{}format\textgreater{}}} 2 bytes]
\textbf{0} = single track file format\\
\textbf{1} = multiple track file format\\
\textbf{2} = multiple song file format (\emph{i.e.}, a series of type 0
files)
\item[\textbf{\texttt{\textless{}n\textgreater{}}} 2 bytes]
number of track chunks that follow the header chunk
\item[\textbf{\texttt{\textless{}division\textgreater{}}} 2 bytes]
unit of time for delta timing. If the value is positive, then it
represents the units per beat. For example, +96 would mean 96 ticks per
beat. If the value is negative, delta times are in SMPTE compatible
units.
\end{description}

\begin{center}\rule{0.5\linewidth}{0.5pt}\end{center}

\protect\hypertarget{track_chunk}{}{}

\hypertarget{track-chunk}{%
\subsection*{Track Chunk}\label{track-chunk}}

A track chunk consists of a literal identifier string, a length
indicator specifying the size of the track, and actual event data making
up the track.

\begin{verbatim}
 
track_chunk = "MTrk" + <length> + <track_event> [+ <track_event> ...]
 
\end{verbatim}

\begin{description}
\tightlist
\item[\textbf{\texttt{"MTrk"}} 4 bytes]
the literal string MTrk. This marks the beginning of a track.
\item[\textbf{\texttt{\textless{}length\textgreater{}}} 4 bytes]
the number of bytes in the track chunk following this number.
\item[\textbf{\texttt{\textless{}track\_event\textgreater{}}}]
a sequenced track event.
\end{description}

\protect\hypertarget{track_event}{}{}

\hypertarget{track-event}{%
\subsection*{Track Event}\label{track-event}}

A track event consists of a delta time since the last event, and one of
three types of events.

\begin{verbatim}
 
track_event = <v_time> + <midi_event> | <meta_event> | <sysex_event>
 
\end{verbatim}

\begin{description}
\tightlist
\item[\textbf{\texttt{\textless{}v\_time\textgreater{}}}]
a variable length value specifying the elapsed time (delta time) from
the previous event to this event.
\item[\textbf{\texttt{\textless{}midi\_event\textgreater{}}}]
any MIDI channel message such as note-on or note-off. Running status is
used in the same manner as it is used between MIDI devices.
\item[\textbf{\texttt{\textless{}meta\_event\textgreater{}}}]
an SMF meta event.
\item[\textbf{\texttt{\textless{}sysex\_event\textgreater{}}}]
an SMF system exclusive event.
\end{description}

\protect\hypertarget{meta_event}{}{}

\hypertarget{meta-event}{%
\subsection*{Meta Event}\label{meta-event}}

Meta events are non-MIDI data of various sorts consisting of a fixed
prefix, type indicator, a length field, and actual event data..

\begin{verbatim}
 
meta_event = 0xFF + <meta_type> + <v_length> + <event_data_bytes>
 
\end{verbatim}

\begin{description}
\tightlist
\item[\textbf{\texttt{\ \textless{}meta\_type\textgreater{}}} 1 byte]
meta event types:

\begin{longtable}[]{@{}llll@{}}
\toprule
\endhead
\textbf{Type} & \textbf{Event} & \textbf{Type} &
\textbf{Event}\tabularnewline
0x00 & Sequence number & 0x20 & MIDI channel prefix
assignment\tabularnewline
0x01 & Text event & 0x2F & End of track\tabularnewline
0x02 & Copyright notice & 0x51 & Tempo setting\tabularnewline
0x03 & Sequence or track name & 0x54 & SMPTE offset\tabularnewline
0x04 & Instrument name & 0x58 & Time signature\tabularnewline
0x05 & Lyric text & 0x59 & Key signature\tabularnewline
0x06 & Marker text & 0x7F & Sequencer specific event\tabularnewline
0x07 & Cue point & &\tabularnewline
\bottomrule
\end{longtable}
\item[\textbf{\texttt{\textless{}v\_length\textgreater{}}}]
length of meta event data expressed as a variable length value.
\item[\textbf{\texttt{\textless{}event\_data\_bytes\textgreater{}}}]
the actual event data.
\end{description}

\protect\hypertarget{sysex_event}{}{}

\hypertarget{system-exclusive-event}{%
\subsection*{System Exclusive Event}\label{system-exclusive-event}}

A system exclusive event can take one of two forms:

\texttt{\ \ sysex\_event\ =\ 0xF0\ +\ \textless{}data\_bytes\textgreater{}\ 0xF7\ }

or

\texttt{\ \ sysex\_event\ =\ 0xF7\ +\ \textless{}data\_bytes\textgreater{}\ 0xF7\ \ }

In the first case, the resultant MIDI data stream would include the
0xF0. In the second case the 0xF0 is omitted.

\begin{center}\rule{0.5\linewidth}{0.5pt}\end{center}

\protect\hypertarget{variable_length}{}{}

\hypertarget{variable-length-values}{%
\subsection*{Variable Length Values}\label{variable-length-values}}

Several different values in SMF events are expressed as variable length
quantities (e.g. delta time values). A variable length value uses a
minimum number of bytes to hold the value, and in most circumstances
this leads to some degree of data compresssion.

A variable length value uses the low order 7 bits of a byte to represent
the value or part of the value. The high order bit is an "escape" or
"continuation" bit. All but the last byte of a variable length value
have the high order bit set. The last byte has the high order bit
cleared. The bytes always appear most significant byte first.

Here are some examples:

\begin{verbatim}
   Variable length              Real value
   0x7F                         127 (0x7F)
   0x81 0x7F                    255 (0xFF)
   0x82 0x80 0x00               32768 (0x8000)
\end{verbatim}

\begin{verbatim}
\end{verbatim}

\begin{center}\rule{0.5\linewidth}{0.5pt}\end{center}

\href{craig@ccrma.stanford.edu}{\texttt{\small craig@ccrma.stanford.edu}}

