\section{\texttt{Distance}}
\label{ugdist}

\texttt{UGens>Filters}

\bigskip

See description appendice\fullref{dist}. 

%%%%%%%%%%%%%%%%%%%%%%%%%%%%%%%%%%%%%%%%%%

\section{\texttt{Doppler4}}
\label{ugdop}

\texttt{UGens>Buffer}

\bigskip

See description on page \pageref{dop}.

%%%%%%%%%%%%%%%%%%%%%%%%%%%%%%%%%%%%%%%%%%

\section{\texttt{Pan4MSXY}}
\label{msxy}

\texttt{UGens>Buffer}

\bigskip

This  \textit{Pseudo-UGen} convert quadraphonic recording MS/XY (using the re- corder Zoom H2N for instance, see page \pageref{mp:msxy} for details) to a four channels equal power panner, with outputs in order LeftFront, RightFront, LeftBack, RightBack.

Note the method \texttt{convertPan4toArray}, which distributes amplitudes in the quadraphonic space according to the panoramic positions (see description on page \pageref{pandist}).

%\newpage
\bigskip

\begin{addmargin}[1em]{0em}%
\mdfdefinestyle{mystyle}{backgroundcolor=gray!5,linecolor=gray!25,roundcorner=3pt}
\begin{mdframed}[style=mystyle]

\bigskip

{\large \textbf{Pan4MSXY}}

\hrulefill

\color{gray!80}Arguments:\color{black} 

\bigskip

\begin{tabular}{l c p{7.5cm}}
\textbf{bufMS} &  & The index of the stereo buffer MS.\\
\textbf{bufXY} &  & The index of the stereo buffer XY.\\
\textbf{dist} & 0 &  0 to +1, relative distance from the listener. \\
\textbf{rate} & 1.0 & speed ratio  of the soundfile. 1.0 is the normal rate, 2.0 is one octave up, 0.5 is one octave down, etc.\\
\textbf{mid} & 1 & as level. \\
\textbf{side} & 1 &  as level.\\
\textbf{xy} & 1 & as level.\\
\textbf{xpos} & 0 &  -1 to +1, x-axis position, \textit{i.e.} left to right.\\
\textbf{ypos} & 0 &  -1 to +1, y-axis position, \textit{i.e.} back to front.%\\
%\textbf{ring} & false & depending of the quadraphonic configuration. If outputs are a ring of channels, swap channels 2 and 3 is required.
\end{tabular}

\bigskip

\end{mdframed}
\end{addmargin}

%%%%%%%%%%%%%%%%%%%%%%%%%%%%%%%%%%%%%%%%%%

\section{\texttt{Sow}}
\label{cfso}

\texttt{UGens>Buffer}

\bigskip

The idea is to apply some `macro' formal objects to a sonic object as it is. These macro-forms are described in the previous chapters respectively as \nameref{imp2} on page \pageref{imp2} and \nameref{imp3} on page \pageref{imp3}.

\bigskip
The sonic object is a sample as a sound file. First, the synthesis algorithm detects the index $ind$ of the absolute maximal value of the signal, in order to compute the duration of the climax of the sample. Then, according to a given array, which is normalised (by divided each element of the array by the smallest value), and which set the ratio of the duplicated sound and the delay to match the climax. For the given array $S$, the ratios $R$ are computed according to two modalities, namely defined by the symbols \texttt{\textbackslash sup} (if we consider the ratio from the biggest value of $S$) or \texttt{\textbackslash inf} (if we consider the ratio from the smallest value of $S$).

\[
     R =
\begin{dcases}
	\frac{min(S)}{(S, >)}  & \text{if \texttt{\textbackslash sup}}\\
	\frac{(S, <)}{min(S) \times max(S)}  & \text{if \texttt{\textbackslash inf}} 
   \end{dcases}
\]

%\newpage
\bigskip
Then, the respective delays $D$ are computed as follow ($sr$ is the sample rate):
 
\[
D = \bigcup_{i \in R} \frac{ind}{sr}  \left( \frac{1}{min(R)} - \frac{1}{R_i} \right)
\]

%%%%%%%%%%%%%%%%%%%%%%%%%%%%%%%%%%%%%%%%%%

\section{\texttt{InH2O}}
\label{inh2o}

\texttt{UGens>Filters}

\bigskip

Simple underwater filter.

\bigskip
\begin{addmargin}[1em]{0em}%
\mdfdefinestyle{mystyle}{backgroundcolor=gray!5,linecolor=gray!25,roundcorner=3pt}
\begin{mdframed}[style=mystyle]

\bigskip

{\large \textbf{InH2O}}

\hrulefill

\color{gray!80}Arguments:\color{black} 

\bigskip

\begin{tabular}{l c p{6cm}}
\textbf{in} &  & The input signal.\\
\textbf{rdepth} & 0.5 & The relative depth, from 0 the surface to 1 (excluded) the deepest point (inaudible).\\
\textbf{turbulance} & \#[0.1, 20] &  Literal array setting the frequency and the level of the modulation of noisy perturbation.\\
\textbf{abs} & -7 & The curvature of the decreasing relative depth.\\
\textbf{numChans} & 2 & The number of output channels. %Most of the time this value should be equal to \texttt{Server.default.options.numOutputBusChannels}.
\end{tabular}

\bigskip

\end{mdframed}
\end{addmargin}

%%%%%%%%%%%%%%%%%%%%%%%%%%%%%%%%%%%%%%%%%%

\section{\texttt{Ulam}}
\label{colz}

\texttt{UGens>Generators>Deterministic}

\bigskip

%\begin{addmargin}[1em]{0em}%
%\mdfdefinestyle{mystyle}{backgroundcolor=gray!5,linecolor=gray!25,roundcorner=3pt}
%\begin{mdframed}[style=mystyle]
%
%\hrulefill
%
%\color{gray!80}Arguments:\color{black} 
%
%\bigskip
%
%\begin{tabular}{l c p{7.5cm}}
%\textbf{ar} &  & Array of integers.\\
%\textbf{stretch} & 5  & Duration of the envelopes.\\
%\textbf{nx} & 0 &  complete the Collatz series for a given size -- i.e. as array method \textsl{clipExtend}. The values under the size of the Collatz sequence returns this sequence as it is.\\
%\textbf{ny} & \textbackslash max &  divide each Collatz serie by the maximal value \textbackslash max - i.e. normalization - or by $n$ \textbackslash freq -- i.e. the value itself as integer.\\
%\textbf{sig} & \textbackslash norm & range the output signal between $-1$ to $1$ with \textbackslash norm, or compute the hyperbolic tangent of the output signal with \textbackslash tanh. 
%\end{tabular}
%
%\bigskip
%
%\end{mdframed}
%\end{addmargin}
%
%\bigskip

See description of the Collatz algorithmic process on page \pageref{colres}. 

See also the class method \texttt{collatz} in the SuperCollider extension \textbf{cycle}. %See also codes \fullref{colzpcsc}.

\href{https://github.com/yannics/cl-mst}{\texttt{\small https://github.com/yannics/cycle}}

\bigskip
For a given positive integer $n$, the Collatz algorithm generate a finite series which is interpreted as an amplitude envelop on the time/x-axis. The envelope is normalised on the amplitude/y-axis by divided the envelope by $n$ as frequency or as a maximal value of the profile. 

\bigskip

Each profile as an envelope is multiplied by its respective \textit{Ugen} taking into account the value $n$ as the frequency argument. This is done recursively through a given array of frequencies as integers.

%%%%%%%%%%%%%%%%%%%%%%%%%%%%%%%%%%%%%%%%%%

