\chapter*{\textsl{Liminaire}}
\phantomsection

\addcontentsline{toc}{chapter}{\textsl{Liminaire}}
\thispagestyle{empty}

\bigskip

With the advent of the internet, sharing and access knowledge have opened a new world of communication toward I hope a new real world. Here is my contribution as a composer of what I call `conceptual music'.						
\bigskip

Conceptual Music consists to highlight and formulate a nodal point of acquired concepts that can be understood in ontological terms. In essence, a concept is the formalisation of structuring object(s) whose emerging phenomenology is either deterministic or indeterministic, and identified as such. My purpose is to point out some nodal points notably through the programming language paradigm as for instance a sketch, a description or a process. This is naturally a work in progress, and the philosophical interpretation of the concept as such remains to be done.						

Or, to put it another way, the definition of the concept in a musical context, consists to pin down the emergent phenomenon as a nodal point of structuring elements. The emergent phenomenon is the sonic object itself that I produce to questioning or integrating my environment as a significative nodal point.

\bigskip
 
The application -- as performances or as scores for instance -- of the said concepts aim to stimulate the imagination and possibly even the creativity of the listener, notably by anamnesis in cognitive terms.										

\bigskip

\begin{center}\rule{0.5\linewidth}{0.5pt}\end{center}

\bigskip
\bigskip

\textbf{Note}: Some codes are stamped \texttt{\textcolor{red}{\small[private]}} because they are too messy to be shared. They will be re-writed and rationalised to be so.
%\newpage

\chapter*{Works}
\phantomsection

\addcontentsline{toc}{chapter}{Works}

%#############################################################################

\section*{Ghost Wind}
\phantomsection

\addcontentsline{toc}{section}{Ghost Wind}
\thispagestyle{empty}

\bigskip

\begin{description}
\item[Concept] \hfill 
\begin{itemize}
\item[] Wind synthesis correlated to a spoken word recording using pitch analysis and formantic analysis with a degree of shape smoothing.
\end{itemize}
\bigskip
\item[Context] \hfill 
\begin{itemize}
\item[] Study in the framework of the acousmatic tale \textit{The Robot And The Baby} conducted by Fr\'{e}d\'{e}ric Voisin in 2012. \\
$\rightarrow$ \href{http://www.fredvoisin.com/spip.php?article215}{\texttt{\small http://www.fredvoisin.com/spip.php?article215}}
\end{itemize}
\bigskip
\item[Required] \hfill 
\begin{itemize}
\setlength\itemsep{1em}
\item[] \texttt{BASH} \\ $\rightarrow$ \href{https://www.gnu.org/software/bash}{\texttt{\small https://www.gnu.org/software/bash}}
\item[] \texttt{PRAAT} \\ $\rightarrow$ \href{http://www.fon.hum.uva.nl/praat}{\texttt{\small http://www.fon.hum.uva.nl/praat}} 
\item[] \texttt{SBCL} \\ $\rightarrow$ \href{http://www.sbcl.org}{\texttt{\small http://www.sbcl.org}}
\end{itemize}
\bigskip
\item[Source] \hfill 
\begin{itemize}
\item[] $\rightarrow$ \href{https://github.com/yannics/GSA/tree/master/SC/ghost-wind}{\texttt{\small https://github.com/yannics/GSA/tree/master/SC/ghost-wind}} 
\end{itemize}
\bigskip
\newpage
\setcounter{footnote}{0}
\item[Alternative] \hfill 
\begin{itemize}
\item[] The \textsl{UGen} \texttt{LPCAnalyzer} with noise as the source -- sounding windy in that way -- uses the linear predictive coding analysis
\citep{mak} %\footnote{\setstretch{0.8}John Makhoul. \textit{Linear Prediction: A Tutorial Review}. Proceedings of the IEEE 63(4), 1975. \\ \href{http://www.ee.iitm.ac.in/~giri/pdfs/reading\_material/Markhoul.pdf}{\scriptsize{\texttt{http://www.ee.iitm.ac.in/$\sim$giri/pdfs/reading\_material/Markhoul.pdf}}} \normalsize{}}
on the input signal, but in this case, the formantic part is eluded because it works only as frequency bandwidths of which the size is determined by the parameter \texttt{noise}. 

\smallskip

\begin{lstlisting}
SynthDef(\LPC, {
  | outBus=0, inBus=0, amp=0, noise=256, 
    xpos=0, ypos=0 |
  Out.ar(outBus, 
    Pan4.ar(
      LPCAnalyzer.ar(
        In.ar(inBus, 1), 
        PinkNoise.ar(0.25), 
        1024, 
        noise), 
      xpos, ypos, amp))
}).add;
\end{lstlisting}
\end{itemize}
\end{description}

\bigskip

\begin{center}\rule{0.5\linewidth}{0.5pt}\end{center}

\bigskip

%#############################################################################

\section*{\texttt{RM236}}
\phantomsection

\addcontentsline{toc}{section}{\texttt{RM236}}

\bigskip

\begin{description}
\item[Concept] \hfill 
\begin{itemize}
\item[] Rhythmic counterpoint -- between 5 complementary rhythms -- for 4 voices and 3 layers, with one as tuning radio sound effects, one as far low bass drums and some kind of high frequencies `sparkles'.
\end{itemize}
\bigskip
\item[Context] \hfill 
\begin{itemize}
\item[] Participation of the \textit{Lake Radio} open call -- \texttt{Works for Radio \#4} -- in 2020. \\
$\rightarrow$ \href{http://thelakeradio.com/call}{\texttt{\small http://thelakeradio.com/call}}
\end{itemize}
\bigskip
\bigskip
\item[Source] \hfill 
\begin{itemize}
\item[] $\rightarrow$ \href{https://github.com/yannics/GSA/tree/master/SC/RM236}{\texttt{\small https://github.com/yannics/GSA/tree/master/SC/RM236}}  
\end{itemize}
\end{description}

\bigskip

\begin{center}\rule{0.5\linewidth}{0.5pt}\end{center}

\bigskip

%#############################################################################

\section*{\textsl{Selenes Havbrev}}
\phantomsection

\addcontentsline{toc}{section}{\textsl{Selenes Havbrev}}

\bigskip

\begin{description}
\item[Concept] \hfill 
\begin{itemize}
\item[] Interactive quadraphonic soundscape using Open Sound Control over WIFI with the modular control surface TouchOSC.
\end{itemize}

\item[Context] \hfill 
\begin{itemize}
\item[] 
See booklet \textsl{Bl\aa stjernehav Familie- \& Dukketeater} on page \pageref{psh}.\\
See also article in \textsl{HAKAPIC, et nettmagasin for kunstkritikk}.\\
$\rightarrow$ \href{https://www.hakapik.no/home/2020/10/8/levende-kunst-i-en-bygning-i-forfall}{\texttt{\scriptsize https://www.hakapik.no/home/2020/10/8/levende-kunst-i-en-bygning-i-forfall}}

\end{itemize}

\item[Required] \hfill 
\begin{itemize}
\setlength\itemsep{1em}
\item[] \texttt{FFmpeg} \\ $\rightarrow$ \href{https://ffmpeg.org/}{\texttt{\small https://ffmpeg.org/}}
\item[] \texttt{SOX} \\ $\rightarrow$ \href{http://sox.sourceforge.net/}{\texttt{\small http://sox.sourceforge.net}}
\item[] \texttt{TouchOSC} \\ $\rightarrow$ \href{https://hexler.net/products/touchosc}{\texttt{\small https://hexler.net/products/touchosc}}
\end{itemize}

\item[Source] \hfill 
\begin{itemize}
\item[] $\rightarrow$ \href{https://github.com/yannics/GSA/tree/master/SC/SelenesHavbrev}{\texttt{\small https://github.com/yannics/GSA/tree/master/SC/SelenesHavbrev}}  -- \texttt{\textcolor{red}{\small[private]}}
\end{itemize}

\item[Notes] \hfill 
\label{mp:msxy}
\begin{enumerate}
\item Quadraphonic XY/MS\\
The recording of the soundscapes is done with the recorder Zoom H2N, which allows to record in 4 channels mode, generating two stereo sound files involving respectively the microphones MS and XY.
 \begin{figure}[H]
\begin{center}
\includegraphics[scale=0.23]{mp/img/H2N.png}
%\caption{Figure on page 21 of the operational manual.}
%\label{h2n}
\end{center}
\end{figure}
\item Decode MS\\
\begin{figure}[!hbt]
	\begin{center}
		\includegraphics[width=33mm]{mp/img/MS}
	\end{center}
\end{figure}\\
\textit{D'origine allemande, MS signifie} \textsl{Mitte Seite} \textit{en Allemand et} \textsl{Middle-Side} \textit{en Anglais.
La radio st\'{e}r\'{e}ophonique transmet le signal sous la forme d'un signal monophonique, et d'un signal de diff\'{e}rence entre canaux, qui s'ajoute \`{a} gauche et se retranche \`{a} droite. La prise de son MS utilise le m\^{e}me principe d\`{e}s la captation. Une capsule cardio\"{i}de ou omnidirectionnelle est point\'{e}e vers le centre de la sc\`{e}ne sonore, une deuxi\`{e}me capsule, \`{a} directivit\'{e} en 8, est plac\'{e} perpendiculairement aussi pr\`{e}s de la premi\`{e}re que possible,
Le passage des canaux gauche et droite aux canaux M et S, et inversement, s'effectue par un proc\'{e}d\'{e} de somme et diff\'{e}rences dit matri\c{c}age:}\\ \\
$M = L + R$\\
$S = L - R$\\
$M + S = ( L + R ) + ( L - R ) = 2L$\\
$M - S = ( L + R ) - ( L - R ) = 2R$\\ \\
\textit{La profondeur de l'effet st\'{e}r\'{e}o se r\`{e}gle facilement en ajustant l'intensit\'{e} relative des deux composantes.}\footnote{\setstretch{0.8}Captation st\'{e}r\'{e}ophonique. \textit{Wikip\'{e}dia, l'encyclop\'{e}die libre.} [Page consult\'{e}e le 31/07/20]\\ \indent \href{http://fr.wikipedia.org/w/index.php?title=Captation\_st\%C3\%A9r\%C3\%A9ophonique}{\scriptsize{\texttt{http://fr.wikipedia.org/w/index.php?title=Captation\_st\%C3\%A9r\%C3\%A9ophonique}}} \normalsize{}}\\
%Manipulating audio
\item Extract audio from MP4\\
\texttt{\scriptsize \$ ffmpeg -i in.mp4 out.wav}
\item Trim audio files\\ 
\texttt{\scriptsize \$ sox initial.wav snippet.wav trim [SECOND TO START] [SECONDS DURATION]}
\item Remove part in audio files\\ 
\texttt{\scriptsize \$ sox in.wav out.wav trim 0 =[SECOND TO START] =[SECOND TO STOP]}
\item TouchOSC setup\\ 
Set iPad TouchOSC \textsf{Layout editor hosts} and \textsf{connections OSC} with the IP of the remote laptop (see \textsf{Network System Preferences}).
\begin{figure}[H]
\hfill \includegraphics[width=0.87\textwidth]{mp/img/ipad1}
\end{figure}
\vspace{-5mm}
Layout used during the play on the 23rd of August 2020 at Troms\o{} Kunstforening.
\begin{figure}[H]
\hfill \includegraphics[width=0.87\textwidth]{mp/img/ipad2}
\end{figure}
\vspace{-5mm}
Layout used during the play on the 10th of October 2020 at Troms\o{} Kunstforening.
\end{enumerate}
\end{description}

On \textcolor{gray}{\large{\ding{202}}} stop and start the system -- boot SC plus start synths --, and on \textcolor{gray}{\large{\ding{203}}} a band reject filter applied to the soundscape according to the actor voices during the play.

\bigskip

\begin{description}
\item[Appendices] \hfill 
\begin{itemize}
\item[$\rightarrow$] Study with vibrating speakers on two recycled corrugated sheets as installation. This installation performed the score of the second guitar of the composition \textsl{Selenes Havbrev} as a MDS -- \textsl{\nameref{mds}} -- with the digital wave guide physical model of a bowed instrument (SuperCollider \textsl{Ugen} \texttt{DWGBowedSimple} part of the SC3plugins) and \texttt{RM236}.\\
\vspace{-4mm}
\begin{figure}[H]
\hfill \includegraphics[width=0.85\textwidth]{mp/img/vs1a}
\end{figure}

\item[$\rightarrow$] Surrealistic sound installation experimenting the SuperCollider \textsl{Ugens} \texttt{DWGBowedSimple} -- playing with the bow parameters such as velocity, force and position --  and \texttt{NTube} (see Acoustics of Tube Models on page \pageref{atm}) -- playing alternatively with the input as a pink noise on two tubes and with the impulse oscillator on ten tubes -- (both are part of the SC3plugins) using respectively vibrating speakers on a cello and on a parabolic antenna, as part of a `post-apocalyptic' ambient quadraphonic soundscape.\\
\vspace{-4mm}
\begin{figure}[H]
\hfill \includegraphics[width=0.87\textwidth]{mp/img/asc}
\end{figure}

\end{itemize}
\end{description}
\vspace{-4mm}


\begin{center}\rule{0.5\linewidth}{0.5pt}\end{center}

\bigskip

%#############################################################################

\section*{Solutions}
\phantomsection
%
\addcontentsline{toc}{section}{Solutions}
%
\vspace{4mm}

\includegraphics[width=\textwidth]{mp/img/S1}

\newpage
\textcolor{white}{...}
\vspace{4mm}

\includegraphics[width=\textwidth]{mp/img/S2}
%\bigskip
%
%\begin{description}
%\item[Concept] \hfill 
%\begin{itemize}
%\item[] ...
%\end{itemize}
%\bigskip
%\item[Context] \hfill 
%\begin{itemize}
%\item[] ...
%\end{itemize}
%\bigskip
%\bigskip
%\item[Source] \hfill 
%\begin{itemize}
$\rightarrow$ \href{https://github.com/yannics/GSA/tree/master/SC/Solutions}{\texttt{\small https://github.com/yannics/GSA/tree/master/SC/Solutions}} 
%\end{itemize}
%\end{description}
%
\bigskip
%
\begin{center}\rule{0.5\linewidth}{0.5pt}\end{center}
%
\bigskip

%#############################################################################

\section*{\texttt{H2O}}
\phantomsection

\addcontentsline{toc}{section}{\texttt{H2O}}
\thispagestyle{empty}

\bigskip

\begin{description}
\item[Concept] \hfill 
	\begin{itemize}
	\item[] The main idea is according to the topic of the ludes, to improvise through the SuperCollider GUI and some kind of live coding through global variables and routines.
	\begin{figure}[H]
	\hfill \includegraphics[width=0.8\textwidth]{mp/img/7589}
	\end{figure}
		\begin{itemize}[leftmargin=0.4in]
		\item \textbf{Prelude} : \textsc{drop} \\ Water drops ambiance playing within the quadraphonic space in terms of density and rhythm with the parameters \texttt{Distance}, \texttt{Tempo}, \texttt{RTM} matrix, and processes of synchronisation and of de-synchronisation.
%Création d'une ambiance de gouttes d'eau dans l'espace en jouant sur la densité et le rythme -- paramètres \texttt{Distance}, \texttt{Tempo}, \texttt{RTM} matrice.
		\item \textbf{Interlude} : \textsc{stream} \\ Water noise from different sources as a background ambiance -- using the \textit{Pseudo-UGen} \texttt{InH2O} (see Section \ref{inh2o}). Some occurrences like the doppler effect on random part of the sample's bank, plus trace of the Prelude in terms of variation playing with \texttt{Tone} and \texttt{Noise}.
%Bruits d'eau de différentes sources en fond sonore avec effet doppler sur des occurrences ponctuelles; plus trace du Prelude en termes de variation en jouant sur la tonalité -- paramètre \texttt{Tone} -- avec plus ou moins de bruit -- paramètre \texttt{Noise}.
		\end{itemize}
	\end{itemize}

\item[Context] \hfill 
	\begin{itemize}
	\item[] Sonic illustration during the storytelling/discussion evening about water, proposed and organised by the \textit{café associatif de Lanester}. This composition for quadraphonic installation was interpreted the 23rd of June, 2023 at the \textit{Tiers-Lieu du Centre Social Albert Jacquard}, Lanester -- \textit{Kreisteiz Breizh}.
	\end{itemize}

\item[Required] \hfill 
\begin{itemize}
\setlength\itemsep{1em}
\item[] \texttt{FluCoMa} \\ $\rightarrow$ \href{https://www.flucoma.org}{\texttt{\small https://www.flucoma.org}}
\end{itemize}

\item[Source] \hfill 
\begin{itemize}
\item[] $\rightarrow$ \href{https://github.com/yannics/GSA/tree/master/SC/H2O}{\texttt{\small https://github.com/yannics/GSA/tree/master/SC/H2O}}  
\end{itemize}

\item[Notes] \hfill 
\begin{enumerate}[label=\alph*)]
\item \textbf{Isolate and analyse water drops}\\ 
Within a bank of samples as the recordings of water drops from home use done by the persons involved in this project, the discrimination is done thanks to the tools of FluCoMa as a SuperCollider extension such as \texttt{FluidBufOnsetSlice}, and the analyse with the combination of  \texttt{FluidBufMFCC} and  \texttt{FluidBufStats}. Then the  \texttt{FluidKDTree} is built from these data in the perspective to apply the method  \texttt{kNearest} as described on the next point about the process of the Prelude.\\ 
The results of these analysing are stored in JSON files respectively as datasets named \textsf{ds\_analysis},  \textsf{ds\_indices}, and \textsf{kdtree}.
\item \textbf{Processing \texttt{kNearest}}\\ 
The function \texttt{$\sim$processNearest} takes three arguments: the index of the rhythm involves (\texttt{RTM1, RTM2, ..., RTM15}), the nearest or the farthest point as boolean, and the length of the result which will be randomly picked (which will assure some kind of disparity and avoid monotony).\\ 
 Then the process goes inside a SC routine triggered from the \texttt{RTM} matrix. Note that the rhythms of the matrix (\texttt{$\sim$rtmBase}) are completely arbitrary.\\ 
 Each position of the drops (and this for each rhythm) -- defined in a panoramic of four channel (see argument \texttt{xpos} and \texttt{ypos} of the \textit{UGen} \texttt{Pan4}) -- is previously randomly computed and stored in an array (\texttt{$\sim$xposAr}, \texttt{$\sim$yposAr}) and can be re-evaluated on the fly, as well as the tempo ratio (\texttt{$\sim$ratAr}).
\end{enumerate}

\end{description}

\titlebox{\textit{\textbf{FluCoMa}}}
{The Fluid Corpus Manipulation project (FluCoMa) instigates new musical ways of exploiting ever-growing banks of sound and gestures within the digital composition process, by bringing breakthroughs of signal decomposition DSP and machine learning to the toolset of techno-fluent computer composers, creative coders and digital artists.
\begin{description}
\item[FluidBufOnsetSlice] \hfill \\ a spectrum-based onset slicers according to a deliberate metric.\\ \href{https://learn.flucoma.org/reference/onsetslice}{\texttt{\small https://learn.flucoma.org/reference/onsetslice}}
\item[FluidBufMFCC] \hfill \\ a classic timbral spectral descriptor, the Mel-Frequency Cepstral Coefficients (MFCCs).\\ \href{https://learn.flucoma.org/reference/mfcc}{\texttt{\small https://learn.flucoma.org/reference/mfcc}}
\item[FluidBufStats] \hfill \\ a statistical analysis on buffer channels such as the mean, the standard deviation, the skewness, the kurtosis, and low, middle, and high as the percentile values of the dataset .\\ \href{https://learn.flucoma.org/reference/bufstats}{\texttt{\small https://learn.flucoma.org/reference/bufstats}}
\item[FluidKDTree] \hfill \\ a k-dimensional tree for efficient neighbourhood searches of multi-dimensional data.\\ \href{https://learn.flucoma.org/reference/kdtree}{\texttt{\small https://learn.flucoma.org/reference/kdtree}}
\end{description}} 

%\vspace{-4mm}


\begin{center}\rule{0.5\linewidth}{0.5pt}\end{center}

\bigskip

%#############################################################################

